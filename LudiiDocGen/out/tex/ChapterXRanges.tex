\chapter{Ranges}  \label{Chapter:Ranges}

In order to simplify the description of {\it ranges} of values, consecutive runs of numbers can be expressed in the form {\tt <int>..<int>} in game descriptions. 
Ranges includes their limits. 
For example, the following ranges:

{\tt
\begin{verbatim}
    7..20
    3..-3
\end{verbatim}
}

\noindent
will expand to these numbers:

{\tt
\begin{verbatim}
    7 8 9 10 11 12 13 14 15 16 17 18 19 20
    3 2 1 0 -1 -2 -3 
\end{verbatim}
}

\noindent
The following range in the game Dash Guti will expand as shown during compilation:

{\tt
\begin{verbatim}
    (place "Counter1" (region {0..9}))
    (place "Counter1" (region {0 1 2 3 4 5 6 7 8 9}))
\end{verbatim}
}

%--------------------------------------------------

\color {gray!95}
{

\section{Smart Ranges}

[FUTURE WORK]

\phantom{}
\noindent
It is possible to also specify ranges based on site coordinates in {\tt String} form, e.g. {\tt "A1".."A12"}.  
If both limits are co-axial then the range will expand consecutive sites along that axis between the specified limits, as follows:

{\tt
\begin{verbatim}
    {"A1".."A12"}
    {"A1" "A2" "A3" "A4" "A5" "A6" "A7" "A8" "A9" "A10" "A11" "A12"'}
    
    {"C2".."F2"}
    {"C2" "D2" "E2" "F2"}
 \end{verbatim}
}

\noindent
Otherwise, the range will expand to all sites within an area delimited by its limits:

{\tt
\begin{verbatim}
    {"B2".."D5"}
 \end{verbatim}
}

\noindent
will expand to:

{\tt
\begin{verbatim}
    {"B2" "B3" "B4" "B5" "C2" "C3" "C4" "C5" "D2" "D3" "D4" "D5"}
 \end{verbatim}
}

}
\color {black}{}
