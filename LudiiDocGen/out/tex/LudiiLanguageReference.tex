\documentclass[10pt,twoside]{report}

\usepackage[table]{xcolor}
\usepackage{mathrsfs,amssymb,amsmath}
\usepackage{graphicx}
\usepackage{hyperref} 
\usepackage{multirow} 
\usepackage{xspace}
\usepackage{booktabs}
\usepackage{enumitem} 
%\usepackage{caption} 
\usepackage{subcaption} 
\usepackage{longtable}
\usepackage[htt]{hyphenat} % to allow the cesure of tt text
\usepackage{dirtree}
\usepackage{eurosym}
\usepackage{float}

% tuning toc, chapters, list items
%\hypersetup{colorlinks,linkcolor={grayred!50!black},citecolor={blue!50!black},urlcolor={blue!80!black}}
\hypersetup{colorlinks,linkcolor={blue!80!black},citecolor={blue!50!black},urlcolor={blue!80!black}}

\usepackage[toc,page]{appendix}
\usepackage[top=1.5in, bottom=1.5in, left=1.41in, right=1.41in]{geometry}
\usepackage{titlesec}
\newcommand{\chapnumfont}{\usefont{T1}{pnc}{b}{n}\fontsize{57}{100}\selectfont}
%\colorlet{chapnumcol}{gray!75}  % color for chapter number
%\colorlet{chapnumcol}{gray!75}  % color for chapter number
%\titleformat{\chapter}[display]{\filleft\bfseries}{\filleft\chapnumfont\textcolor{chapnumcol}{\thechapter}}{-24pt}{\Huge}
\titleformat{\chapter}[display]{\filleft\bfseries}{\filleft\chapnumfont{\thechapter}}{-24pt}{\Huge}
\setlist{topsep=2.2pt,itemsep=0.5pt} %nolistsep}%\setlist[itemize]{itemsep=0.5pt}
\setcounter{secnumdepth}{3}
\setcounter{tocdepth}{3} 

% patch below because lost numbering of sections
\usepackage{etoolbox}
\makeatletter
\patchcmd{\ttlh@hang}{\parindent\z@}{\parindent\z@\leavevmode}{}{}
\patchcmd{\ttlh@hang}{\noindent}{}{}{}
\makeatother

\usepackage{tikz}
\usetikzlibrary{shapes,calc,positioning,automata,arrows,trees}
\usepackage[tikz]{bclogo}
\renewcommand\logowidth{15pt}
\newcommand\bcpen{\includegraphics[width=\logowidth]{figs/bluePen.png}} 
\newcommand\bcdico{\includegraphics[width=\logowidth]{figs/yellowBook.png}} 
\newcommand\bcroue{\includegraphics[width=\logowidth]{figs/orangeWheel.png}} 
\newcommand\bcludii{\includegraphics[width=\logowidth]{figs/ludii-icon-2.pdf}} 
\newcommand\bcjava{\includegraphics[width=\logowidth]{figs/java-icon-2b.pdf}} 
\newcommand\bcgrammar{\includegraphics[width=\logowidth]{figs/folder-icon-3.png}} 
\usepackage[skins,breakable,xparse]{tcolorbox}
\tikzset{
  dirtree/.style={
    grow via three points={one child at (0.8,-0.7) and two children at (0.8,-0.7) and (0.8,-1.45)}, 
    edge from parent path={($(\tikzparentnode\tikzparentanchor)+(.4cm,0cm)$) |- (\tikzchildnode\tikzchildanchor)}, growth parent anchor=west, parent anchor=south west},
}

\usepackage{listingsutf8}

\definecolor{v2lgray}{gray}{0.85}
\definecolor{dgray}{rgb}{0.4,0.4,0.4}
\definecolor{dblue}{RGB}{0,0,99}
\definecolor{dred}{RGB}{150,6,54}
\definecolor{dgreen}{RGB}{47,135,7}
\definecolor{dviolet}{RGB}{102,0,153}
\definecolor{mblue}{RGB}{0,0,180}
\definecolor{colorja}{RGB}{255,248,220}
\definecolor{colorex}{HTML}{DFEFFF}  %{FFE3BE}
\definecolor{colorsy}{HTML}{F2F2F2}
\definecolor{grey1}{rgb}{0.9,0.9,0.9}

\newcommand{\bnf}[1]{\textsl{\color{dblue}{#1}}}
\newcommand{\bnfX}[1]{\texttt{<}\bnf{#1}\texttt{.../>}}
\newcommand{\norX}[1]{\texttt{<#1.../>}}

\newcommand{\refalgorithm}[1]{Algorithm~\ref{#1}}
\newcommand{\refappendix}[1]{Appendix~\ref{#1}}
\newcommand{\refchapter}[1]{Chapter~\ref{#1}}
\newcommand{\refequation}[1]{\eqref{#1}}
\newcommand{\reffigure}[1]{\figurename~\ref{#1}}
\newcommand{\refsection}[1]{Section~\ref{#1}}
\newcommand{\refsubsection}[1]{Subsection~\ref{#1}}
\newcommand{\reftable}[1]{Table~\ref{#1}}   
\newcommand{\refpart}[1]{Part~\ref{#1}}

\newcommand{\inlinefigure}[1]{
    \begingroup
    \setbox0=\hbox{\includegraphics[height=\baselineskip]{#1}}%
    \parbox{\wd0}{\box0}\endgroup
}

\usepackage[export]{adjustbox}
\usepackage{cleveref}
\usepackage{subcaption}
\usepackage{apacite}

\usepackage{longtable}

\let\counterwithout\relax
\let\counterwithin\relax
\usepackage{chngcntr}
\counterwithout{footnote}{chapter}

\usepackage{listingsutf8}

\lstdefinelanguage{ludii}{
  keywords={game,metadata,control,mode,rules,keyword,text,time,play,player,equipment,board,square,moves,to,byPiece,slide,indexOf,empty,end,line,result,index,stalemated,then,replay,even,if,turn,shoot,lastToMove,
  get,item,component,container,cross,disc,ball,queen,dot,colour,hand,owner,numItems,modify,wheel,hexHex,rect,spokes,aligned,dim,cols,join,cut,remove,cellA,in,mover,option,HexBoard,regions,edge,connect,define,step,isFriend,pawn,
  cellB,list,end,start,place,who,count,target,posn,fromTo,or,action,regionFunction,sites,resultType,line,not,occupied,store,length,add,top,bottom
},
  basewidth  = {.6em,0.6em},
  keywordstyle=\color{mblue}\bfseries,
  ndkeywords={Alternating,Discrete,Real,Mover,Win,Lose,Loss,Draw,Tie,Abort,None,Any,All,Each,In,Out,Along,CW,Vert,Horz,F,B,L,R,N,E,S,W,U,D,P1,P2,P3,P4,NE,SW,NW,SE,ForwardDiagonal,
  P5,P6,P7,P8,Opposite,Next,Prev,Odd,Even,Empty,Own,Enemy,Ally,Partner},
  ndkeywordstyle=\color{dviolet}\bfseries,
  identifierstyle=\color{black},
  sensitive=true,   % need case-sensitivity for different keywords
  comment=[l]{//},
  morecomment=[s]{<!--}{-->},
  commentstyle=\color{dred}\ttfamily,
  stringstyle=\color{dgreen}\ttfamily,
  morestring=[b]',
  morestring=[b]",
  escapechar=@,
  showstringspaces=false,
  xleftmargin=1pt,xrightmargin=1pt,
  breaklines=true,basicstyle=\ttfamily\small,backgroundcolor=\color{colorex},inputencoding=utf8/latin9,texcl
}

\lstdefinelanguage{syntax}{
  keywords={game,metadata,control,rules,keyword,text,time,play,player,equipment,board,square,moves,to,indexOf,empty,end,line,result,index,mode,modeType,compassDirection,tiling,directions,directionChoice,playout,model,addToEmpty,byPiecePlayout,
  get,item,component,container,disc,ball,colour,hand,owner,numItems,modify,wheel,hexHex,rect,spokes,aligned,dim,cols,join,cut,remove,cellA,
  cellB,list,end,start,place,who,count,target,posn,fromTo,or,action,regionFunction,sites,resultType,line,not,occupied,store,where,what,class,
  arg,subClass,terminal,add,roleType,knight,cross,bishop,king,queen,pawn,rook,piece,card,die,letter,number,tile,chess,timeType,basis,label,
  rows,players,name,and,dirn,dirnType,length,sitesFrom,sitesTo,functions,cont,site,role,num,types,cell,alternatingMove,realTime,simultaneousMove,map,regions,track,deck,dominoes,cardType,byPiece,byPieceType,captureByApproach,captureByWithdraw,checkmate,checkMove,hop,leap,pass,promotion,select,shoot,slide,step,constraints,flanked,flip,forDirn,if,observe,pending,priority,replay,roll,setCounter,setDouble,setOwner,setScore,setState,sow,surrounded,then,top,variableType,region,included,exception,recursive,allSites,around,regionType,borderRegion,bottom,centre,column,corners,difference,directionRegion,edge,edgeFace,edgeVertex,exteriors,facesEdge,hintRegion,own,phase,playable,row,set,stateRegion,union,verticeFace,walk,direction,stepType,other,hint,values,puzzleConstraints,allDifferent,excepts,except,connected,areaType,crossing,equalParities,forEach,mult,shape,shapeType,solved,sum,unique,decision,decisionPuzzle,large,domino,largePiece,lPiece,animal,arrow,checkers,dominoHalf,dot,hands,hex,pill,senetPiece,shogi,pieceState,stratego,tafl,triangle,urPiece,xiangqi,arrow,camel,cat,cow,dog,elephant,goat,horse,leopard,rabbit,sheep,tiger,wolf,dame,man,manStar,mann,paper,rock,scissors,fuhyo,ginsho,hisha,kakugyo,keima,kinsho,kyosha,narigin,narikei,narikyo,osho,ryuma,ryuo,tokin,state,flipDisc,realTimePiece,real_time,bike,bomb,captain,colonel,flag,general,lieutenant,major,marshal,miner,scout,sergeant,spy,jarl,thrall,jiang,ju,ma,pao,shi,xiang,zu,alquerqueBoard,arimaaBoard,backgammonBoard,boardless,cardBoard,chessBoard,chineseCheckersBoard,connect4Board,goBoard,graphBoard,halmaBoard,hexBoard,hexYBoard,mancalaBoard,morrisBoard,peralikatumaBoard,reversiBoard,senetBoard,shogiBoard,snakesAndLaddersBoard,solitaireBoard,strategoBoard,surakartaBoard,taflBoard,urBoard,wheelBoard,xiangqiBoard,puzzleBoard,joinDiago,joinOrtho,modifyType,cutAll,cutDiago,cutOrtho,futoshikiBoard,graphPuzzleBoard,kakuroBoard,lineSegmentBoard,nonogramBoard,regionPuzzleBoard,sudokuBoard,dotBoard,hashiBoard,handDice,startRule,deal,fill,fogOfWar,hints,initScore,placeRandomly,setCount,byScore,abs,counter,directionSite,double,forwardOnTrack,from,height,lastEdgeMove,lastFromMove,lastToMove,level,max,min,mod,mover,next,posPiece,previous,replayCount,score,size,turn,value,le,adjacent,allPass,canMove,configuration,connect,contains,encircle,equal,even,full,ge,gt,in,isCheckmate,isEnemy,isFriend,isFriendAt,isMover,isNext,isPending,isPiece,isPrev,isState,know,le,lt,odd,reachedRegion,stalemated,threatened,visited,sub
},
  keywordstyle=\color{dblue}\slshape,ndkeywords={String,int,boolean,string,ints,booleans,integer,Integer,Boolean},
  ndkeywordstyle=\color{dviolet}\bfseries,
  basewidth  = {.5em,0.5em},
  escapechar=@,
  xleftmargin=1pt,xrightmargin=1pt,
  breaklines=true,basicstyle=\ttfamily\linespread{1.0}\small,backgroundcolor=\color{colorsy},inputencoding=utf8/latin9,texcl
}

\newcommand{\core}[1]{ 
  \medskip \begin{tcolorbox}[
    enhanced,breakable,
    boxsep=0pt,top=0pt,bottom=0pt,left=7mm,right=1mm,
    toprule=0.1mm,leftrule=0.1mm,rightrule=0.25mm,bottomrule=0.25mm,shadow={0.2mm}{-0.2mm}{0mm}{dgray},
    overlay unbroken and first={\node (logo) at ([xshift=6mm,yshift=-5mm]frame.north west) {#1}; %\draw[black,line width=1.5pt] (logo) -- ([xshift=4mm,yshift=1.5mm]frame.south west);  
    },
    colframe=dgray,titlerule=-0.2mm,toptitle=3mm,coltitle=black,fonttitle=\bfseries,
    lines before break=6, pad at break*=10pt
}

\newcounter{cntEx}

%\def\bw{1}
%\ifx\bw\undefined
  \lstnewenvironment{ludii}[1][]{\lstset{language=ludii,#1}}{} 
  \newenvironment{boxex}
    {\stepcounter{cntEx} \core{\bcludii} ,colback=colorex,title style={color=colorex},title=~ Ludeme \thecntEx]}
    {\end{tcolorbox}} %{\vspace{-0.1cm}\end{tcolorbox} ~ \vspace{-0.2cm}}

\usepackage{multicol}
\setlength\emergencystretch{4em}\hbadness=10000

% Thanks for this to: https://tex.stackexchange.com/a/331796/121370
\newenvironment{ttquote}
 {\list{}{%
    \ttfamily
    \setlength{\itemindent}{-1em}%
    \setlength{\listparindent}{\itemindent}%
    \setlength{\rightmargin}{\leftmargin}%
    \setlength{\parsep}{0pt plus 1pt}%
  }%
  \item[]}
 {\endlist}

\newtcolorbox{formatbox}{colback=gray!8,breakable=true,frame hidden,boxrule=0pt,enhanced jigsaw,left=0pt,bottom=0pt}

\usepackage{fancyhdr}
\pagestyle{fancy}
\fancyhf{}
\renewcommand{\headrulewidth}{0pt}
\renewcommand{\chaptermark}[1]{\markboth{#1}{}}
\renewcommand{\sectionmark}[1]{\markright{#1}}

%=======================================================

\begin{document}

\texttt{\hyphenchar\font=-1 }

\thispagestyle{empty}

\begin{centering}

\includegraphics[scale=0.35]{figs/ludii-icon-1.pdf}

 %-------- Title --------

\vspace{30mm}
\noindent\rule{14.5cm}{0.5pt}

\vspace{5mm}
{\Huge \bf Ludii Language Reference}

\vspace{2mm}
\noindent\rule{14.5cm}{0.5pt}
 
%-------- Authors -------- 
 
\vspace{10mm}
{\Large Cameron Browne, Dennis J. N. J. Soemers,\\ {\'E}ric Piette, Matthew Stephenson and Walter Crist}

\vspace{10mm}
{\large Department of Data Science and Knowledge Engineering (DKE)}
 
\vspace{1mm}
{\large Maastricht University}

\vspace{1mm}
{\large Maastricht, the Netherlands}

\vspace{10mm}
{\large Ludii Version 1.3.4}

\vspace{1mm}
{\large \today}

\pagebreak
%\vspace{30mm}
%\includegraphics[scale=0.3]{figs/LOGO_ERC-FLAG_EU_.jpg}

\end{centering}

\begin{centering}

%==================================================================

\thispagestyle{empty}

{\Large \bf Ludii Language Reference}

\end{centering}

\phantom{}

\noindent
This document provides full documentation for the game description language used by the Ludii general game system. Note that the majority of this document is automatically generated. 

The source code of Ludii is available at \url{https://github.com/Ludeme/Ludii}.

More info on Ludii may be found on its website: \url{https://ludii.games/}. For questions or suggestions, please contact us on the Ludii forums (\url{https://ludii.games/forums/}), or send an email to \href{mailto:ludii.games@gmail.com}{ludii.games@gmail.com}.


%==================================================================

{
\hypersetup{linkcolor={gray!50!black}}
\tableofcontents
}

%==================================================================

\chapter{Introduction}

\fancyhead[LE]{\thepage}
\fancyhead[RO]{\thepage}
\fancyhead[C]{\leftmark {} - \rightmark}
\fancyhead[LO]{Language Reference}
\fancyhead[RE]{Ludii}

This document provides a full reference for the game description language used to describe games for the Ludii general game system. Games in Ludii are described as {\it ludemes}, which may intuitively be understood to encapsulate simple concepts related to game rules or equipment. Every game description starts with a {\tt game} ludeme, described in the form \texttt{(game ...)} in game description files. Here, the dots (\texttt{...}) are a placeholder for one or more arguments that are supplied to the \texttt{game} ludeme. 

Arguments provided to ludemes may be:
\begin{itemize}
\item {\it Strings}: described in \refsection{Sec:Introduction.Strings}.
\item {\it Booleans}: described in \refsection{Sec:Introduction.Booleans}.
\item {\it Integers}: described in \refsection{Sec:Introduction.Integers}.
\item {\it Floats}: described in \refsection{Sec:Introduction.Floats}.
\item {\it Other ludemes}: described throughout most of the other chapters of this document.
\end{itemize}

\refpart{Part:GameDescriptions} describes all the ludemes that can be used in game descriptions. This is the most important part for writing new games that can be run in Ludii. \refpart{Part:Metadata} describes ludemes that can be used to add extra metadata to games. These are not strictly required for games to run, but can be used to provide additional information about games in Ludii, or to modify how they look in Ludii or how Ludii's AIs play them. More advanced language features are described in \refpart{Part:MetalanguageFeatures}. 

%-----------------------------------------------------------------
% WARNING: Don't change this label without modifying our Java code,
% we generate references to this label from Java!!!!
\section{Strings} \label{Sec:Introduction.Strings}
%-----------------------------------------------------------------

Strings are simply snippets of text, typically used to assign names to pieces of game equipment, rules, or other concepts. Strings in game descriptions can be written by wrapping any snippet of text in a pair of double quotes. For instance, \texttt{"Pawn"} can be used to provide a name to a piece. By convention, the first symbol in a string is usually an uppercase character, but this is generally not required.

%-----------------------------------------------------------------
% WARNING: Don't change this label without modifying our Java code,
% we generate references to this label from Java!!!!
\section{Booleans} \label{Sec:Introduction.Booleans}
%-----------------------------------------------------------------

There are two boolean values; \texttt{true} and \texttt{false}. They can be written as such in any game description file, without any additional notation.

%-----------------------------------------------------------------
% WARNING: Don't change this label without modifying our Java code,
% we generate references to this label from Java!!!!
\section{Integers} \label{Sec:Introduction.Integers}
%-----------------------------------------------------------------

Integers are numbers without a decimal component, such as \texttt{1}, \texttt{-1}, \texttt{100}, etc. They can simply be written as such, without any additional notation, in Ludii's game description language.

%-----------------------------------------------------------------
% WARNING: Don't change this label without modifying our Java code,
% we generate references to this label from Java!!!!
\section{Floats} \label{Sec:Introduction.Floats}
%-----------------------------------------------------------------

Floating point values are numbers with a decimal component, such as \texttt{0.5}, \texttt{-1.2}, \texttt{5.5}, etc. If a ludeme expects a floating point value as an argument, it must always be written to include a dot. For example, \texttt{1} cannot be interpreted as a floating point value, but \texttt{1.0} can.

%=============================  PART I  ===============================

\part{Ludemes} \label{Part:GameDescriptions}

\include{Chapter2GameLudemes}
\include{Chapter3Equipment}
\include{Chapter4GraphFuncs}
\include{Chapter5DimFuncs}
\include{Chapter6FloatFuncs}
\include{Chapter7Rules}
\include{Chapter8Moves}
\include{Chapter9BooleanFuncs}
\include{Chapter10IntFuncs}
\include{Chapter11IntArrayFuncs}
\include{Chapter12RegionFuncs}
\include{Chapter13DirectionFuncs}
\include{Chapter14RangeFuncs}
\include{Chapter15Utilities}
\include{Chapter16Types}

%=============================  PART II  ===============================

\part{Metadata} \label{Part:Metadata}

\include{Chapter17InfoMetadata}
\include{Chapter18GraphicsMetadata}
\include{Chapter19AIMetadata}

%=============================  PART III  ================================

\part{Metalanguage Features} \label{Part:MetalanguageFeatures}

\chapter{Defines}  \label{Chapter:Defines}

The {\tt define} is a mechanism for replacing text in game descriptions with simple short labels, much like {\it macros} are used in programming languages. 
Defines are the first of several Ludii metalanguage features intended to make game descriptions clearer and more powerful.

\phantom{}
\noindent
Defines are useful for:
\begin{itemize}
\item Simplifying game descriptions by wrapping complex ludeme structures into simple labels.
\item Giving meaningful names to useful game concepts.
\item Gathering repeated ludeme structures that are duplicated across multiple games into a single reusable definition. 
\end{itemize}

\phantom{}
\noindent
Defines can occur anywhere in the game description file -- even within ludemes -- but are typically at the top of the file, before the {\tt game} reference, to impart some structure. 
Defines can be nested within other defines... but not themselves! 
The convention is to name each {\tt define} with a single compound word in ``UpperCamelCase'' format.

%--------------------------------------------------------------------------------

\section{Example}

For example, the game description for Breakthrough contains the following {\tt define}:

{\tt
\begin{verbatim}
    (define "ReachedTarget" (in (lastTo) (region Mover)) )
\end{verbatim}
}

This {\tt define} wraps up the concept the current mover's last ``to'' move landing in the target region with the label ``ReachedTarget''. 
The game's {\it end} rule can then be simplified and clarified as follows:

{\tt
\begin{verbatim}
    (end (if ("ReachedTarget") (result Mover Win)))
\end{verbatim}
}

This concept is used in many games, all of which can reuse this {\tt define} to simplify and clarify their own descriptions.

%--------------------------------------------------------------------------------

\section{Parameters}

Ludii {\tt define}s can be parameterised for greater flexibility. 
The parameters to be passed in to a define can take the following form:

{\tt
\begin{verbatim}
    keyword
    (clause ...)
    name:keyword
    name:(clause ...)
\end{verbatim}
}

Parameters are matched to {\it insertion points} of the form $\#N$ within the define, where $N$ is the number of the parameter to be instantiated. 
For example, the following define:

{\tt
\begin{verbatim}
    (define "Outcome" (result #1 #2))
\end{verbatim}
}

\noindent
can be instantiated with any of the following calls:

{\tt
\begin{verbatim}
    ("Outcome" Mover Win)    
    ("Outcome" (next) Lose)  
    ("Outcome" All Draw)    
\end{verbatim}
}

\noindent
to give:

{\tt
\begin{verbatim}
    (result Mover Win)    
    (result (next) Lose)  
    (result All Draw)    
\end{verbatim}
}

Parameterised {\tt define}s {\it must} be surrounded by brackets that enclose the label and its parameters when instantiated. 
Defines can contain arbitrary text, but should have balanced brackets, i.e. the same number of open and close brackets `(' for `)' and `\{' for `\}'. 
Non-parameterised {\tt define}s do not need to be bracketed, but it is recommended to do so for consistency and readability. 
Both of the following formats are allowed but the first format is preferred:

{\tt
\begin{verbatim}
    (end (if ("ReachedTarget") (result Mover Win)))
    (end (if  "ReachedTarget"  (result Mover Win)))
\end{verbatim}
}

%--------------------------------------------------------------------------------

\section{Null Parameters}

Sometimes a {\tt define} might be useful but its parameters do not match the current circumstance. 
In this case, a null parameter placeholder character '~' may be passed instead of that parameter, which simply instantiates to nothing.
Null parameters make {\tt define}s even more powerful, by allowing the same {\tt define} to be used in different ways by different games.
For example, the following {\tt define}:

{\tt
\begin{verbatim}
    (define "HopSequenceCapture" 
        (hop
            (between #1 #2 
                if:(isEnemy (who at:(between))) 
                (apply (remove (between) #3))
            )
            (to if:(in (to) (empty)))
            (consequence 
                (if (canMove 
                        (hop 
                            (from (lastTo)) 
                            (between #1 #2
                                if:(and (not (in (between) (sitesToClear))) 
                                   (isEnemy (who at:(between))))
                                (apply (remove (between) #3))
                            )
                            (to if:(in (to) (empty)))
                        )
                    ) 
                    (moveAgain)
                )
            )   
        )  
    )  
\end{verbatim}
}

\noindent
is called as follows in the game Coyote:
 
{\tt
\begin{verbatim}
    ("HopSequenceCapture" ~ ~ at:EndOfTurn) 
\end{verbatim}
}

The first two parameters are null placeholders, so all occurrences of ``\#1'' and ``\#2'' in the ``HopSequenceCapture'' {\tt define} will be instantiated with the empty string ``'', while all occurrences of ``\#3'' will be instantiated with ``at:EndOfTurn''.

%--------------------------------------------------------------------------------

\section{Known Defines}

External {\tt define}s called {\it known defines} can also be called within a game description simply by invoking their names (with suitable parameters). 
Each such known {\tt define} must: 
\begin{itemize}
\item be declared in a file with the same name as its label, 
\item have the file extension *.def, and 
\item be located in Ludii's "def" folder (or below it). 
\end{itemize}

\phantom{}
\noindent
The list of known defines provided with the Ludii distribution is given in Appendix~\ref{Appendix:KnownDefines}. \\
In addition, each game will typically have a known {\tt ai} metadata entry of the form:

{\tt
\begin{verbatim}
    (metadata
        ...
        (ai 
            "Chess_ai"
        )
    )    
\end{verbatim}
}

This is a reference to the known {\tt ai} define that is automatically generated for each game, which stores the relevant AI settings for that game and its various options.
Details of the  {\tt ai} metadata format are given in Chapter~\ref{Chapter:AIMetadata}.


\chapter{Options}  \label{Chapter:Options}

For many games there exist alternative rule sets and other variable aspects, such as different board sizes, number of pieces, starting positions, and so on. 
The Ludii game compiler supports an {\it option} mechanism to allow such alternatives for a game to be defined in a single description, to avoid the need to implement each one in its own file. 
Options are defined outside the main {\tt game} ludeme but typically used within it. 
Options are typically declared directly below the {\tt game ...)} ludeme, for clarity.

Options are instantiated at compile time and can be arbitrarily large, including choices between complete game descriptions if desired. 
Multiple options can be specified in combination, to give dozens or even hundreds of variant rule sets for a single game description.

%-------------------------------------------------------------------------------------

\section{Syntax}

Each set of options is declared with the {\tt option} keyword and constitutes an option {\it category} with a number of option {\it items}. 
Each option item has one or more named {\it arguments}.  
Option are described as follows:

{\tt
\begin{verbatim}
    (option "Heading <Tag> args:{ <argA> <argB> <argC> ... } {
        (item "Item X" <Xa> <Xb> <Xc> ... "Description of item X.")  
        (item "Item Y" <Ya> <Yb> <Yc> ... "Description of item Y.")****   
        (item "Item Z" <Za> <Zb> <Zc> ... "Description of item Z.")*   
        ...
    })
\end{verbatim}
}

\noindent
where:
\begin{itemize}
\item {\tt "Category A"} is the category name.
\item {\tt Tag} is a tag used to locate the position in the game description where the option is to be instantiated.
\item {\tt argA/B/C} are named arguments for each item.
\item {\tt "Item X/Y/Z"} are the item names.
\item {\tt Xa, Xb, Xc} are the actual option arguments to instantiate.
\item {\tt "Description of item X/Y/Z."} describe each item in user friendly terms.
\end{itemize}

\phantom{}
\noindent
Option items are referenced in the game description by tag-argument pairs {\tt <Tag:arg>}. \\
For example, this option call in the game description:

{\tt
\begin{verbatim}
    (ludeme <Tag:argB>)
\end{verbatim}
}

\noindent
would be instantiated as follows if the user selects the menu item ``Category A/Item Y":

{\tt
\begin{verbatim}
    (ludeme Yb)
\end{verbatim}
}

%-------------------------------------------------------------------------------------

\section{Option Priority}

A number of asterisks may optionally be appended to the end of each option item. 
The number of asterisks indicate that item's {\it priority} rating, with a higher number meaning higher priority. 

If no user-selected options are specified when a game is compiled, then the highest priority item within each category becomes the current option for that category. 
If more than one item exists with the highest priority rating, then the first item listed with this rating is chosen. 
For example, ``Item Y" would be the highest priority item for ``Category A'', with priority rating ****, and the default option to be instantiated in the absence of any user-selected options.

%-------------------------------------------------------------------------------------

\section{Example}

The following example shows the option mechanism in action for the game of Hex. 
The game description assumes the existence of two option categories -- {\tt Board} and {\tt Result} -- with item arguments {\tt size} and {\tt type}, respectively.  

{\tt
\begin{verbatim}
    (game "Hex"  
        (players 2)  
        (equipment { 
            (board (hex Diamond <Board:size>)) 
            (piece "Ball" Each)
            (regions P1 { (sites Side NE) (sites Side SW) })
            (regions P2 { (sites Side NW) (sites Side SE) })
        })  
        (rules 
            (meta (swap))
            (play (move Add (to (sites Empty)))) 
            (end (if (is Connected Mover) (result Mover <Result:type>))) 
        )
    )
\end{verbatim}
}

\noindent
The two option categories are declared as follows. 
Note that the 11x11 option has the highest priority, while the 10x10, 14x14 and 17x17 options are next priority below. 
These are the most common sizes of Hex boards, so are the most interesting options for the user.

{\tt
\begin{verbatim}
    (option "Board Size" <Board> args:{ <size> } {
        (item   "3x3"  <3> "The game is played on a 3x3 board.")   
        (item   "4x4"  <4> "The game is played on a 4x4 board.")   
        (item   "5x5"  <5> "The game is played on a 5x5 board.")   
        (item   "6x6"  <6> "The game is played on a 6x6 board.")   
        (item   "7x7"  <7> "The game is played on a 7x7 board.")   
        (item   "8x8"  <8> "The game is played on a 8x8 board.")   
        (item   "9x9"  <9> "The game is played on a 9x9 board.")   
        (item "10x10" <10> "The game is played on a 10x10 board.")*   
        (item "11x11" <11> "The game is played on a 11x11 board.")****   
        (item "12x12" <12> "The game is played on a 12x12 board.")   
        (item "13x13" <13> "The game is played on a 13x13 board.")   
        (item "14x14" <14> "The game is played on a 14x14 board.")*   
        (item "15x15" <15> "The game is played on a 15x15 board.")   
        (item "16x16" <16> "The game is played on a 16x16 board.")   
        (item "17x17" <17> "The game is played on a 17x17 board.")*   
        (item "18x18" <18> "The game is played on a 18x18 board.")   
        (item "19x19" <19> "The game is played on a 19x19 board.")   
    })

    (option "End Rules" <Result> args:{ <type> } {
        (item "Standard"  <Win>  "The first player to connect his two sides wins.")*   
        (item "Misere"    <Loss> "The first player to connect his two sides loses.")   
    })
\end{verbatim}
}

\noindent
If the user selects the ``Board Size/9x9" and ``End Rules $>$ Misere'' menu item, then the game will be instantiated as follows during compilation, to give a {\it mis\`{e}re} version of the game on a 9$\times$9 board:

{\tt
\begin{verbatim}
    (game "Hex"  
        (players 2)  
        (equipment { 
            (board (rhombus 9)) 
            (piece "Ball" Each)
            (regions P1 { (sites Side NE) (sites Side SW) })
            (regions P2 { (sites Side NW) (sites Side SE) })
        })  
        (rules 
            (meta (swap) )
            (play (add (empty))) 
            (end (if (isConnected Mover) (result Mover Loss))) 
        )
    )
\end{verbatim}
}


\chapter{Rulesets}  \label{Chapter:Rulesets}

In addition to the {\tt option} mechanism described in the previous chapter, the Ludii game compiler also supports a {\tt ruleset} mechanism that allows user to declare custom rule sets defined by combinations of options. 
User defined rulesets are of the form:

{\tt
\begin{verbatim}
    (rulesets { 
        (ruleset "Ruleset/Name A" { "Option A/Item M" "Option B/Item N" ...})
        (ruleset "Ruleset/Name B" { "Option C/Item O" "Option D/Item P" ...})*
        ...
    })
\end{verbatim}
}

\noindent
where:
\begin{itemize}
\item {\tt "Ruleset/"} denotes this as a ``Ruleset'' menu item.
\item {\tt "Name A/B"} are unique user-specified names for each ruleset.
\item {\tt "Option A/Item M",  "Option B/Item N", ...} are the actual options that make up this ruleset, as declared in their respective menu items.
\end{itemize}

Rulesets have a similar priority rating mechanism to options, i.e. rulesets with more asterisks appended to their declaration are deemed higher priority.

%-------------------------------------------------------------------------------------

\section{Example}

The following example shows the ruleset mechanism in action for the game of Seega. 
This game description has a single option category -- {\tt Board}  -- with two item arguments {\tt size} and {\tt numInitPiece}.  

{\tt
\begin{verbatim}
    (game "Seega"  
        (players 2)  
        (equipment { (board (square <Board:size>)) ... }) 
        (rules (start (place "Ball" "Hand" count:<Board:numPieces>)) ...)
    )

    (option "Board Size" <Board> args:{ <size> <numPieces>} {
        (item "5x5" <5> <12> "The game is played on a 5x5 board.")**   
        (item "7x7" <7> <24> "The game is played on a 7x7 board.")   
        (item "9x9" <9> <40> "The game is played on a 9x9 board.")   
    })

    (rulesets { 
        (ruleset "Ruleset/Khamsawee" { "Board Size/5x5" })*
        (ruleset "Ruleset/Sebawee"   { "Board Size/7x7" })
        (ruleset "Ruleset/Tisawee"   { "Board Size/9x9" })
    })
\end{verbatim}
}

\noindent
If the user selects ``Ruleset/Sebawee'' from the menu, then its option ``Board Size/7x7'' will be instantiated to give:

{\tt
\begin{verbatim}
    (game "Seega"  
        (players 2)  
        (equipment { (board (square 7)) ... }) 
        (rules (start (place "Ball" "Hand" count:24)) ...)
    )
\end{verbatim}
}

\noindent
If no user-selected ruleset is specified, then the game is compiled with the highest priority ruleset by default.

\chapter{Ranges}  \label{Chapter:Ranges}

In order to simplify the description of {\it ranges} of values, consecutive runs of numbers can be expressed in the form {\tt <int>..<int>} in game descriptions. 
Ranges includes their limits. 
For example, the following ranges:

{\tt
\begin{verbatim}
    7..20
    3..-3
\end{verbatim}
}

\noindent
will expand to these numbers:

{\tt
\begin{verbatim}
    7 8 9 10 11 12 13 14 15 16 17 18 19 20
    3 2 1 0 -1 -2 -3 
\end{verbatim}
}

\noindent
The following range in the game Dash Guti will expand as shown during compilation:

{\tt
\begin{verbatim}
    (place "Counter1" (region {0..9}))
    (place "Counter1" (region {0 1 2 3 4 5 6 7 8 9}))
\end{verbatim}
}

%--------------------------------------------------

\color {gray!95}
{

\section{Smart Ranges}

[FUTURE WORK]

\phantom{}
\noindent
It is possible to also specify ranges based on site coordinates in {\tt String} form, e.g. {\tt "A1".."A12"}.  
If both limits are co-axial then the range will expand consecutive sites along that axis between the specified limits, as follows:

{\tt
\begin{verbatim}
    {"A1".."A12"}
    {"A1" "A2" "A3" "A4" "A5" "A6" "A7" "A8" "A9" "A10" "A11" "A12"'}
    
    {"C2".."F2"}
    {"C2" "D2" "E2" "F2"}
 \end{verbatim}
}

\noindent
Otherwise, the range will expand to all sites within an area delimited by its limits:

{\tt
\begin{verbatim}
    {"B2".."D5"}
 \end{verbatim}
}

\noindent
will expand to:

{\tt
\begin{verbatim}
    {"B2" "B3" "B4" "B5" "C2" "C3" "C4" "C5" "D2" "D3" "D4" "D5"}
 \end{verbatim}
}

}
\color {black}{}

\chapter{Constants}  \label{Chapter:Constants}


A number of pre-defined constants can be used in game descriptions. 
These are instantiated with their actual values at compile time.

%-------------------------------------------------------------------------------------

\section{Off}

Denotes an off-board position.

\phantom{}
\noindent
Internal value: -1 

\vspace{1mm}
\subsubsection*{Example}
\begin{formatbox}
\noindent\begin{minipage}{\textwidth}
\begin{verbatim}
(not (= (where "King" Next) Off))
\end{verbatim}
%\vspace{.1mm}
\end{minipage}
\end{formatbox}

%-------------------------------------------------------------------------------------

\section{End}

Denotes the end of a track.

\phantom{}
\noindent
Internal value: -2 

\vspace{1mm}
\subsubsection*{Example}
\begin{formatbox}
\noindent\begin{minipage}{\textwidth}
\begin{verbatim}
(track "Track1" {0..5 7..12 25..20 18..13 End} P1 directed:true)
\end{verbatim}
%\vspace{.1mm}
\end{minipage}
\end{formatbox}

%-------------------------------------------------------------------------------------

\section{Undefined}

Denotes a general "undefined" condition, for example if the game logic queries the value of a site that is out of range of the board.

\phantom{}
\noindent
Internal value: -1 

%-------------------------------------------------------------------------------------



%=====================================================================

\chapter*{Acknowledgements}

This work is part of the {\it Digital Ludeme Project}, funded by \euro2m European Research Council (ERC) Consolidator Grant \#771292 being run by Cameron Browne at Maastricht University's Department of Data Science and Knowledge Engineering over 2018--23. 

The Ludii team consists of Cameron Browne (Principal Investigator), {\'E}ric Piette, Matthew Stephenson and Walter Christ (Postdoctoral Researchers) and Dennis Soemers (PhD Candidate). 
We thank Tahmina Begum and Wijnand Engelkes for contributions to this document.

\includegraphics[scale=0.3,right]{figs/LOGO_ERC-FLAG_EU_.jpg}

%==================================================================

\appendix

\include{AppendixAImageList}

\fancyhead[C]{\leftmark {} - \rightmark}

\include{AppendixBKnownDefines}

\include{AppendixCLudiiGrammar}

%===============================================================

\end{document}
